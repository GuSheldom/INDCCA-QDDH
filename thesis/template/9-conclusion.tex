\chapter{Conclusion and Future work}\label{chap:conclusion}


\section{Conclusion}
\label{sec:conclusion}

Upon reflection, the formalization of algebraic adversaries and oracle simulation in EasyCrypt we introduced 
not only used for verifying the IND-CCA1 security of ElGamal, but also offers a general and reusable basic structure for future theoretical analysis for Algebraic Group Model (AGM). 

The core operator we introduced, \texttt{prodEx}, with its supporting lemmas on distributivity, scaling, and vector operations, enables almost all manipulation of exponent-linear relations within EasyCrypt, give an essential capability for reasoning about algebraic adversaries in a mechanized setting.

Additionally, this framework can be directly extended to formalize and verify other results from the AGM paper by Fuchsbauer, Kiltz, and Loss~\cite{fuchsbauer2018}. 
For instance, their proof of the \emph{Algebraic Computational Diffie--Hellman (aCDH)} assumption is equivalent to the standard \emph{Computational Diffie--Hellman (CDH)} assumption relies on representing every group element as a linear combination of previously known elements. 
In their paper proofs, this reasoning is expressed informally through algebraic extraction arguments. 
In the future analysis, by using our \texttt{prodEx} operator, such relations can be explicitly encoded as:
\[
h = \mathsf{prodEx}([g_1, g_2, \ldots, g_n], [a_1, a_2, \ldots, a_n]),
\]
which allows EasyCrypt to mechanically verify whether the adversary’s output \(h\) follows from known bases. 
This transformation turns the intuitive statement 'since the adversary is algebraic, we can extract its coefficients' in the original paper into a formally verified lemma within EasyCrypt.




\section{Future Directions}
\label{sec:future}
Looking forward, this approach provides several promising research directions. 
First, it could support the development of a \textbf{mechanized AGM lemma library}, covering results such as a CDH--CDH equivalence, algebraic SDH reductions~\cite{sdh2020}, and algebraic LRSW assumptions, providing a reusable foundation for future mechanized cryptographic proofs. Additionally, by combining these algebraic representations with probabilistic reasoning in EasyCrypt, it might be possible to make proof on \textbf{hybrid models} that mix algebraic reasoning with post-quantum assumptions (e.g., MLWE~\cite{nguyen2022} or isogeny-based systems~\cite{de2017}). 

Such extensions would not only strengthen the theoretical rigor of the AGM but also provide a concrete bridge between algebraic reasoning and post-quantum formal verification. All in all, our work make contribution to the long-term goal of creating a unified, machine-verified code approach for a accurate and reproducible cryptographic security reductions.
