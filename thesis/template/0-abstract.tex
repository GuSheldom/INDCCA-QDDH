\chapter*{Abstract}
Public-key cryptography forms the foundation of modern secure digital communication, yet the informal nature of many security proofs can make them difficult to verify or susceptible to errors. This thesis presents a machine-checked formal proof, indicating its bilateral equivalence between the \INDCCAone security of \ElGamal based Key Encapsulaton Mechanism(\KEM) and the \qDDH assumption.





Unlike the standard \ElGamal public-key encryption, which is only \INDCCAcpa secure, the \KEM formulation encapsulates a session key derived from the \ElGamal game, allowing a stronger \INDCCAone guarantee. Both directions of the reduction, from an \INDCCAone adversary to a \qDDH distinguisher and the reverse are mechanised in EasyCrypt, eliminating hidden assumptions and ensuring proof soundness.






 


Methodologically, we introduce a reusable operator-and-lemma library that represents the exponent linear algebra and its faithful reflection in the group, comprising operators such as \texttt{prodEx}, \texttt{addv}, \texttt{scalev}, \texttt{sumv}, and \texttt{shift\_trunc}. This library provides a principled and verifiable oracle-simulation framework that captures the behavior of algebraic adversaries.

Keywords: bilateral equivalence, \qDDH, \INDCCAone, \ElGamal, \KEM, EasyCrypt, Algebraic Group Model, formal verification, tight reduction
