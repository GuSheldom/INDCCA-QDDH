
% define your own macros here

\newcommand{\Eff} {\ensuremath{\mathit{eff}}}  % example command without arguments
\newcommand{\Pre} {\ensuremath{\mathit{pre}}}  % (again)

% --- Common notation used in this thesis ---
% Sets, fields, groups
\newcommand{\Group}{\ensuremath{\mathcal{G}}}
\newcommand{\Zp}{\ensuremath{\mathbb{Z}_p}}

% Problems / security notions / schemes
\newcommand{\INDCCAone}{\ensuremath{\mathsf{IND\text{-}CCA1}}\xspace}
\newcommand{\INDCCAcpa}{\ensuremath{\mathsf{IND\text{-}CPA}}\xspace}

\newcommand{\qDDH}{\ensuremath{\mathsf{q\text{-}DDH}}\xspace}
\newcommand{\DDH}{\ensuremath{\mathsf{DDH}}\xspace}
\newcommand{\KEM}{\ensuremath{\mathsf{KEM}}\xspace}
\newcommand{\ElGamal}{\ensuremath{\mathsf{ElGamal}}\xspace}

% Advantage and probability
\newcommand{\Adv}[2]{\ensuremath{\mathsf{Adv}^{#1}_{#2}}}
\newcommand{\Prob}[1]{\ensuremath{\Pr\!\left[\,#1\,\right]}}

% Algorithms and operators
\newcommand{\Enc}{\ensuremath{\mathsf{Enc}}\xspace}
\newcommand{\Dec}{\ensuremath{\mathsf{Dec}}\xspace}
\newcommand{\Gen}{\ensuremath{\mathsf{Gen}}\xspace}
\newcommand{\getsr}{\ensuremath{\xleftarrow{\$}}} % random sampling

% Linear-algebra helpers used in AGM-style proofs
\newcommand{\prodEx}{\ensuremath{\mathsf{prodEx}}}
\newcommand{\addv}{\ensuremath{\mathsf{addv}}}
\newcommand{\scalev}{\ensuremath{\mathsf{scalev}}}
\newcommand{\sumv}{\ensuremath{\mathsf{sumv}}}
\newcommand{\shifttrunc}{\ensuremath{\mathsf{shift\_trunc}}}
\newcommand{\zerov}{\ensuremath{\mathsf{zerov}}}

% Adversaries
\newcommand{\Aalg}{\ensuremath{\mathcal{A}}}
\newcommand{\Balg}{\ensuremath{\mathcal{B}}}

% Note that you can easily specify arguments:
% \newcommand{\someMacro}[2] {Argument 1: #1, Argument 2: #2} % example command with two arguments
% you use it via \someMacro{Hello}{World!}


% the following commands are being provided by the amsthm package
% the first parameter states the new environmet's name that can be
% used (due to this definition here) and the second the name that
% will appear in the PDF document
\theoremstyle{definition}
\newtheorem{definition}{Definition}   % well, a formal definition!
\theoremstyle{plain}
\newtheorem{prop}{Proposition} % like a theorem, but less important or evolved
\newtheorem{assumption}{Assumption} % for assumptions like AGM

\newtheorem{lem}{Lemma}        % used within a proof of a theorem
\newtheorem{thm}{Theorem}      % well, a theorem! :) important and evolved
\newtheorem{cor}{Corollary}    % basically either a proposition or theorem,
                               %  but one that follows from another theorem.


% Aliases to support common environment names used in text
\newtheorem{lemma}[lem]{Lemma}
\newtheorem{theorem}[thm]{Theorem}
\newtheorem{proposition}[prop]{Proposition}
\newtheorem{corollary}[cor]{Corollary}                               
% There's a lot you can configure about the appearance. If interested,
% open the manual of amsthm or google for tutorials etc. on that package

% the following add a symbol to the definition environment to make it more
% clear when a definition ends (as there is no difference in fonts!). From:
% https://tex.stackexchange.com/questions/226334/change-a-amsthm-theorem-ending
\newcommand{\xqed}[1]{%
    \leavevmode\unskip\penalty9999 \hbox{}\nobreak\hfill
    \quad\hbox{\ensuremath{#1}}}
\newcommand{\Endofdef}{\xqed{\blacksquare}}
\newenvironment{defn}[1]{%
    \begin{definition}#1}{%
    \Endofdef\end{definition}%
}
