\chapter{Related Work}\label{chap:relatedWork}

IND-CCA (Indistinguishability under Chosen Ciphertext Attack) security is a core concept in public-key cryptography. It requires that even when attackers can query a decryption oracle, they still cannot distinguish the target plaintext, thus providing one of the strongest confidentiality guarantees for encryption schemes. IND-CCA security proofs are often complex and tedious, and traditional manual reduction processes are prone to errors. To reduce this risk, the academic community has increasingly focused on formal verification tools such as EasyCrypt in recent years~\cite{easycrypt}. Barthe et al. first conducted mechanized verification of some conclusions that were previously only argued intuitively in the Generic Group Model (GGM). Furthermore, EasyCrypt has been used for formal verification of modern encryption schemes, such as the IND-CCA security of Kyber during the NIST standardization process, whose proof was completed through machine verification using the Fujisaki-Okamoto transformation under the random oracle model~\cite{kyber2024}.

Nevertheless, EasyCrypt also has limitations. Particularly, the library and ecosystem are still relatively underdeveloped, resulted in many commonly required basic lemmas and operators are not able to used directly, meaning that users are often required to implement basic algebraic machinery themselves in order to complete non-trivial proofs. This lack of readily reusable components increases the effort, but it also motivates the systematic development of reusable operators and lemmas, enriching EasyCrypt's ecosystem for future proof.

In this context, researchers have proposed some alternative models to obtain tighter or more concise security proofs. The most representative is the Algebraic Group Model (AGM) proposed by Fuchsbauer, Kiltz, and Loss~\cite{fuchsbauer2018}. In AGM, it is assumed that adversaries are algebraic, meaning that whenever they output a new group element, they must simultaneously provide a representation of how this element is formed by combining previous elements. This restriction provides additional structural information for reductions, making the proof process more concise and tight. A key result by the authors (Theorem 5.2) shows that breaking the IND-CCA1 security of ElGamal is tightly equivalent to solving the q-type DDH problem under AGM. This proof uses the algebraic representations provided by adversaries to construct consistent decryption oracle simulation, achieving an almost lossless security reduction.

The advantage of the AGM approach is that it provides tight reductions (no loss in attack advantage) and allows reasoning to avoid the complex guessing and technical rewinding common in standard model proofs. However, its limitations are also obvious: AGM relies on non-standard assumptions about adversaries, and existing proofs are still manual derivations that have not undergone mechanized verification. Although these proofs are highly convincing, some steps are still based on intuition rather than formal derivation, so potential subtle errors may still exist, which is precisely where formal methods can provide compensation.

Beyond encryption schemes, AGM has also been applied to more complex cryptographic protocols. In their 2023 work "From Polynomial IOP and Commitments to Non-malleable zkSNARKs," Faonio et al.~\cite{faonio2023} used AGM (combined with the random oracle model) to prove the simulation-extractability of the KZG polynomial commitment scheme, thereby constructing non-malleable zkSNARKs. This shows that AGM is also valuable in security analysis of complex protocols, especially in scenarios where direct proofs are difficult to provide in the standard model. However, similar to the ElGamal case, this work still remains at the manual proof level and has not achieved mechanized verification. This also indicates that there is still a gap between AGM-style reasoning and fully formal verification. How to combine intuitive algebraic reasoning with machine-verifiable formal methods remains an important challenge facing this field.

When it comes to the the ElGamal-based Key Encapsulation Mechanism (\KEM), which has been widely studied under various computational assumptions and application settings. Galindo et al.~\cite{galindo2016} introduced a leakage-resilient variant of the ElGamal KEM, named BEG-KEM, which enhances resistance against side-channel attacks through masking and blinding techniques while maintaining practical performance on embedded platforms. Hashimoto et al.~\cite{hashimoto2022} further strengthened the theoretical foundations by providing a tight reduction proof for the Twin-DH hashed ElGamal KEM in multi-user environments, demonstrating its scalability and security. Earlier, Kiltz and Pietrzak~\cite{kiltz2010} established one of the first leakage-resilient ElGamal constructions, proving that a CCA1-secure KEM combined with a one-time symmetric cipher can yield full \INDCCAone security even under bounded key leakage. Finally, Lee et al.~\cite{lee2023} introduced a decomposable KEM framework that achieves continuous key agreement with reduced bandwidth while retaining DDH-based security guarantees. 
Together, these works demonstrate the evolution of ElGamal-based KEMs from theoretical constructs toward efficient, leakage-resilient, and tightly secure primitives across a variety of cryptographic contexts.
